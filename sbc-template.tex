\documentclass[12pt]{article}

\usepackage{sbc-template}

\usepackage{graphicx,url}

%\usepackage[brazil]{babel}   
\usepackage[latin1]{inputenc}  
     
\sloppy

\title{Estudo de Caso da infraestrutura de banco de dados da Uber}

\author{Athos Castro Moreno\inst{1}, Lais Brazil Peixoto Macedo\inst{1} }

\address{Departamento de Computa��o - Universidade Tecnol�gica Federal do Paran�
  (UFPR)\\
  86300000 -- Corn�lio Proc�pio -- PR -- Brazil
  \email{athos@alunos.utfpr.edu.br, lais.macedo@outlook.com}
}

\begin{document} 

\maketitle
     
\begin{resumo} 
  
\end{resumo}

\section{Introdu��o}

Com o crescimento extraordin�rio das metr�poles mundiais, segundo a GSMA (Entidade Global de Telefonia M�vel) existem 5 bilh�es de pessoas no mundo com Smartphones.....

\section{Problema} 

A dificuldade para conseguir servi�os como T�xi e entregas no tempo que voc� quer e daquilo que voc� precisa tem atrasado milh�es de pessoas. Considerando que o n�mero de pessoas conectadas e com um aparelho m�vel...

\section{Objetivo}

\section{Estudo de Caso}

\subsection{Empresa estudada} % Lais

A empresa norte-americana UberCab, fundada em 2009 por Garrett Camp e Travis Kalanick com a proposta inicial de ser um servi�o semelhante a um T�xi com carros de Luxo pretos tendo as requisi��es feitas por um dispositivo eletr�nico. Em Maio de 2010 foi lan�ada a vers�o Beta e somente em 2011 oficialmente os servi�os e aplicativos Mobile para IOS e Android na cidade de S�o Francisco com pre�os mais caros que um T�xi normal. Em 2011 a companhia mudou seu nome para Uber e em Julho de 2012 foi introduzido o servi�o UberX permitindo que pessoas dirigissem com seus pr�prios carros e em 2013 o servi�o j� estava em 35 cidades. Hoje a Uber possui servi�os de transportes, entregas de comidas e pacotes, com colaboradores utilizando desde bicicletas at� caminh�es, presente em 633 cidades do mundo.
 
\subsection{Tecnologia utilizada} % Athos

\subsection{T�cnicas de arquitetura de software utilizadas} % Lais, microservi�os

\subsubsection{Revis�o sobre os conceitos} % Athos

\subsection{Detalhes da Implementa��o}

\subsection{Resultados}

\section{An�lise cr�tica sobre o estudo de caso}

\section{Li��es aprendidas}

\section{Conclus�o}

\section{Referencias}

\bibliographystyle{sbc}
\bibliography{sbc-template}

http://exame.abril.com.br/tecnologia/5-bilhoes-de-pessoas-tem-smartphones/
https://www.uber.com/pt-BR/our-story/

\end{document}
