\documentclass[12pt]{article}

\usepackage{sbc-template}

\usepackage{graphicx,url}
\usepackage[alf]{abntex2cite}

\usepackage[brazil]{babel}   
\usepackage[latin1]{inputenc} 
     
\sloppy

\title{Estudo de Caso da infraestrutura de banco de dados da Uber}

\author{Athos Castro Moreno\inst{1}, Lais Brazil Peixoto Macedo\inst{1} }

\address{Departamento de Computa��o - Universidade Tecnol�gica Federal do Paran�
  (UTFPR)\\
  Avenida Alberto Carazzai, 1640 -- 86300000 -- Corn�lio Proc�pio -- PR -- Brasil
  \email{athos@alunos.utfpr.edu.br, lais.macedo@outlook.com}
}

\begin{document} 

\maketitle
     
\begin{resumo} 
  
\end{resumo}

\section{Introdu��o}


\section{Problema} 

\section{Objetivo}

\section{Estudo de Caso}

\subsection{Empresa estudada} % Lais

A empresa norte-americana UberCab, fundada em 2009 por Garrett Camp e Travis Kalanick com 
a proposta inicial de ser um servi�o semelhante a um t�xi com carros de luxo pretos tendo as 
requisi��es feitas por um dispositivo eletr�nico. 
Em maio de 2010 foi lan�ada a vers�o beta e somente em 2011 o servi�o foi lan�ado oficialmente
 na cidade de S�o Francisco com pre�os mais caros que um T�xi normal. 
Em 2011 a companhia mudou seu nome para Uber e em Julho de 2012 foi introduzido 
o servi�o UberX permitindo que pessoas dirigissem com seus pr�prios carros e em 2013 o servi�o j� estava em 35 cidades. 
Hoje a Uber possui servi�os de transportes, entregas de comidas e pacotes, 
com colaboradores utilizando desde bicicletas at� caminh�es, presente em 633 cidades do mundo \cite{uber0000}.

\subsection{Tecnologia utilizada} % Athos

\subsection{T�cnicas de arquitetura de software utilizadas} % Lais, microservi�os

\subsubsection{Revis�o sobre os conceitos} % Athos

\subsection{Detalhes da Implementa��o}

\subsection{Resultados}

\section{An�lise cr�tica sobre o estudo de caso}

\section{Li��es aprendidas}

\section{Conclus�o}

\bibliography{bibliografia}

\end{document}
